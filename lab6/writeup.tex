% template created by: Russell Haering. arr. Joseph Crop
\documentclass[12pt,letterpaper]{article}
\usepackage{anysize}
\marginsize{2cm}{2cm}{1cm}{1cm}

\begin{document}

\begin{titlepage}
    \vspace*{4cm}
    \begin{flushright}
    {\huge
        ECE 375 Lab 6\\[1cm]
    }
    {\large
        Remotely Operated Vehicle
    }
    \end{flushright}
    \begin{flushleft}
    Lab Time: Wednesday 5-7
    \end{flushleft}
    \begin{flushright}
    Sean Rettig
    \vfill
    \rule{5in}{.5mm}\\
    TA Signature
    \end{flushright}

\end{titlepage}

\section{Introduction}

The purpose of this fifth lab was to familiarize ourselves with the concepts
related to using USART I/O and further improve our knowledge of interrupts.

\section{Program Overview}

\subsection{Remote}

The remote program sends commands to the robot via the USART infrared LED.  For
each command, it first sends a robot ID that is shared between the remote and
the robot.  The robot should only respond to commands preceded by the correct
ID.  The most significant bit of each ID is 0, while the most significant bit
of each command is 1, allowing the types of data to be differentiated.  Since
commands store information in the upper 4 bits of the byte, they are shifted
down so the most significant bit of each command can be a 1; the robot unshifts
the command before running it.

\subsubsection{Initialization Routine}

First, USART1 is initialized to run at 2400 baud with 8 data bits and 2 stop
bits; the robot is set the same way.  It also enables transmitting.

\subsubsection{Main Routine}

The main routine simply sends a list of predefined commands to the robot in a
loop, waiting some number of seconds in between each command.

\subsubsection{Command Routines}

There is a command routine for each of the 5 possible commands (move forward,
move backward, turn left, turn right, and halt).  Each one simply sends the
robot ID byte followed by the command byte.

\subsubsection{Wait Routine}

The Wait routine simply runs a long loop during which it does nothing in order
to "wait" a certain amount of time.

\subsection{Robot}

The robot waits until it 

\subsubsection{Initialization Routine}

First, the I/O ports are initialized so the motors and whiskers can be used
(just like in the BumpBot program).  Then, USART1 is initialized to run at 2400
baud with 8 data bits and 2 stop bits; the remote is set the same way.  It also
enables receiving and receive interrupts.  Interrupts for the whiskers are also
set.

\subsubsection{Main Routine}

The main routine does nothing but loop infinitely, as all actions to be
performed run on interrupts.

\subsubsection{USART\_Receive Routine}

This is the real "main" routine; it runs whenever USART data has been received
and processes it.  It first checks if the data is an ID or a command; if it's
an ID, it checks if it's the correct one, and if so, goes into "accept mode",
meaning that it will run the next command given to it.  If it's given a
command, it will run it only if the robot is already in accept mode, and then
exit accept mode since it had just accepted a command; the next command
received from a remote will be preceded by its own ID byte.

\subsubsection{HitRight Routine}

The HitRight Routine is also triggered by interrupts and runs when the right
whisker is hit.  When this routine runs, it ignores any remote commands given
to it while it moves backward for 1 second and turns left for 1 second.  It
will then continue whatever it what doing before the routine was triggered.  It
accomplishes the timed events by setting movement signals to PORTB and then
calling the Wait routine to wait a short amount of time before setting a new
movement signal.

\subsubsection{HitLeft Routine}

The HitLeft Routine is also triggered by interrupts and runs when the left
whisker is hit.  When this routine runs, it ignores any remote commands given
to it while it moves backward for 1 second and turns right for 1 second.  It
will then continue whatever it what doing before the routine was triggered.  It
accomplishes the timed events by setting movement signals to PORTB and then
calling the Wait routine to wait a short amount of time before setting a new
movement signal.

\subsubsection{Wait Routine}

The Wait routine simply runs a long loop during which it does nothing in order
to "wait" a certain amount of time.

\section{Conclusion}

In this lab, I learned about how to initialize and use USART for both input and
output, as well as trigger interrupts on USART events.  While the code for this
lab now seems fairly simple, getting it working for the first time was very
confusing as there is a lot that can go wrong, given how much configuration is
needed to get USART, the interrupts, and the ports working not only
individually, but also between the AVRs.

\section{Source Code}

\subsection{Remote}

\begin{verbatim}
\end{verbatim}

\subsection{Robot}

\begin{verbatim}
\end{verbatim}

\end{document}

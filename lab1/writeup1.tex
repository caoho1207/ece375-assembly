% template created by: Russell Haering. arr. Joseph Crop
\documentclass[12pt,letterpaper]{article}
\usepackage{anysize}
\usepackage{enumerate}
\marginsize{2cm}{2cm}{1cm}{1cm}

\begin{document}

\begin{titlepage}
    \vspace*{4cm}
    \begin{flushright}
    {\huge
        ECE 375 Lab 1\\[1cm]
    }
    {\large
        Introduction to AVR Tools
    }
    \end{flushright}
    \begin{flushleft}
    Lab Time: Wednesday 5-7
    \end{flushleft}
    \begin{flushright}
    Sean Rettig
    \vfill
    \rule{5in}{.5mm}\\
    TA Signature
    \end{flushright}

\end{titlepage}

\section{Additional Questions}
\begin{enumerate}

    \item Go to the lab webpage and download the template write-up. Read it
        thoroughly and get familiar with the expected format.  Specifically
        look at the included source code. What type of font is used? What size
        is the font? From here on when you include your source code in your lab
        write-up you must adhere to that font type and size.

        The text of the answer

    \item Take a look at the code you downloaded for today’s lab. Notice the
        lines that begin with .def and .equ followed by some type of
        expression. These are known as pre-compiler directives. Define
        pre-compiler directive. What is the difference between the .def and
        .equ directives (HINT: see section 5.1 of the AVR Starter Guide given
        on the lab webpage).

        answer

    \item Take another look at the code you downloaded for today’s lab. Read
        the comment that describes the macro definitions. From that explanation
        determine the 8-bit binary value of the following expressions. Note:
        the numbers below are decimal values.

        \begin{enumerate}[a.]

            \item $(1<<2)$
            
                answer

            \item $(1<<2)$
            
                answer

            \item $(4>>1)$
                
                answer

            \item $(1<<4)$
            
                answer

            \item $(6>>1|1<<6)$
            
                answer

        \end{enumerate}

\end{enumerate}

\end{document}

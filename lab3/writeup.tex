% template created by: Russell Haering. arr. Joseph Crop
\documentclass[12pt,letterpaper]{article}
\usepackage{anysize}
\marginsize{2cm}{2cm}{1cm}{1cm}

\begin{document}

\begin{titlepage}
    \vspace*{4cm}
    \begin{flushright}
    {\huge
        ECE 375 Lab 3\\[1cm]
    }
    {\large
        Data Manipulation and the LCD Display
    }
    \end{flushright}
    \begin{flushleft}
    Lab Time: Wednesday 5-7
    \end{flushleft}
    \begin{flushright}
    Sean Rettig
    David Winkler
    \vfill
    \rule{5in}{.5mm}\\
    TA Signature
    \end{flushright}

\end{titlepage}

\section{Introduction}

The purpose of the third lab was not only to acquaint ourselves with the concepts associated with using the AVR's LCD display, but also programming the AVR in general using assembly, being the first time we have done so.  In particular, we learned about initializing the stack pointer and copying data from program memory to data memory.

\section{Program Overview}

This is a fairly simple program that just loads two predefined strings and displays them on the LCD display.

\section{Initialization Routine}

The stack pointer is first initialized so that that stack can be used.  Then the LCD display initialization subroutine is called, and our two strings are transferred from program memory to data memory in two separate loops.  Each loop first loads the program memory address into the Z register and the data memory address that the LCD screen reads from into the Y register.  Then the data is transferred from where Z points to where Y points byte-by-byte using an intermediate register and post-incrementing Z and Y.  This continues until the entire string has been copied.

\section{Additional Questions}
\begin{enumerate}

    \item question

        answer

\end{enumerate}

\section{Conclusion}

\section{Source Code}

\begin{verbatim}
\end{verbatim}

\pagebreak

\section{Challenge Source Code}

\begin{verbatim}
\end{verbatim}

\end{document}
